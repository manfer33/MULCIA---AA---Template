\documentclass{mulcia_aa}

\name{Nombre}
\lastName{Apellidos}

\begin{document}
\genTitle
\genAdvice

\begin{problem}
Sea $X$ un universo y $D$ un conjunto de entrenamiento sobre $X$. Sean $E^+ = \{ x \in X |(x, 1) \in D \}$ y $E^- = \{x \in X |(x, 0) \in D\}$. Sea $H$ el conjunto de hipótesis que contiene a todas las hipótesis sobre $X$. Sea $VS$ el espacio de versiones para $ \langle D, H \rangle$. Sea $x_0 \in X$ tal que $x_0 \not\in E^+ \cup E^- $ y sea $h \in VS$ tal que $h(x_0) = 0$. Demostrar que existe $h' \in VS$ tal que $h'(x_0) = 1$.
\end{problem}
\begin{solution}
Solución
\end{solution}

\begin{problem}
Sea $U$ un universo finito y $C = 2^U$ el conjunto de los objetivos. Sea $\mathcal{H}$ un conjunto de hipótesis sobre $U$ y $L$ un algoritmo de aprendizaje tal que su dominio es $\bigcup_{c \in C}\bigcup_{m \geq 1} \mathcal{S}(m, c)$. Demostrar que si $\mathcal{H} \not= 2^U$ , entonces $L$ no es consistente.
\end{problem}
\begin{solution}
Solución
\end{solution}

\begin{problem}
Sea $D = \{\langle x_1, c(x_1)\rangle, . . . ,\langle x_n, c(x_n)\rangle\}$ un conjunto de entrenamiento para un concepto $\mathcal{C}$ y sea $\mathcal{H}$ un conjunto de hipótesis. Demostrar que el resultado de aplicar el algoritmo de ELIMINACIÓN DE CANDIDATOS es el mismo para cualquier ordenación de los elementos de $D$.
\end{problem}
\begin{solution}
Solución
\end{solution}
 
\begin{problem}
Aplica los algoritmos de aprendizaje \emph{por enumeración} y  \emph{Find-S} para los siguientes problemas de aprendizaje:
\end{problem}
\vspace{-5pt}
\begin{solution}

\begin{itemize}
    \item \textbf{Problema 1} \\
    $X = \mathbb{R}^2$ \\
    $H = \{h_n:X\to \{0,1\}|n\in\mathbb{N}\wedge h_n((x,y))=1\Leftrightarrow x^2+y^2 \le n^2 \}$ \\
    $s = \{\langle (1, 1),1\rangle,\langle (3, 4),1\rangle,\langle (2, 2),1\rangle,\langle (4, 7),0\rangle\}$\\
    
    \item \textbf{Problema 2} \\
    $X = \mathbb{R}^2$ \\
    $H = \{h_n|n\in \mathbb{N}\}$ con $h_0=\emptyset$ y si $n >1$ entonces \\
    \textcolor{white}{.}$\ \ \ \ \ \ \ h_n=\{(x,y)\in X|a,b\in\mathbb{N}, a\le x < b, n = \frac{b(b-1)}{2}+a+1\}$\\
    $s = \{\langle (0, 0),0\rangle,\langle (3, 4),1\rangle,\langle (2, 2),1\rangle\}$\\
\end{itemize}
\end{solution}

\begin{problem}
En este ejercicio consideraremos $X = \{0, 1\}^n$, i.e., $X$ es el conjunto de todas las cadenas de longitud $n$ formadas por ceros y unos.
\end{problem}
\begin{solution}
    \begin{enumerate}
        \item ¿Cuantos ejemplos positivos del concepto palíndromo hay en $X$?\\
        
        Solución    
        
        \item Sea $\omega$ el concepto definido en $X$ de la siguiente manera: $\omega(y) = 1$ si y sólo si $y$ contiene como máximo dos 1’s. Prueba que el número de ejemplos positivos de $\omega$ es una función cuadrática de $n$.\\
        
        Solución    
        
        \item Supongamos que en un problemas de aprendizaje sobre $X$ aplicamos el algoritmo de aprendizaje por enumeración sobre el conjunto de todas las hipótesis y las hipótesis están enumeradas de manera que la que buscamos está en la primera mitad. Si podemos probar un millón de hipótesis por segundo y $X = \{0, 1\}^9$, ¿cuánto tiempo llevará encontrar la hipótesis buscada en el peor de los casos?\\
        
        Solución
    \end{enumerate}
\end{solution}
\end{document}
